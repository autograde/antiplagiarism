\documentclass[12pt]{article}

\usepackage{amssymb}
\usepackage{booktabs}
\usepackage{floatrow}
\usepackage{graphicx}
\usepackage{listings}
\usepackage{url}

\title{Anti-plagiarism readme}
\date{2015-10-21}
\author{Eric Scott Freeman}

\begin{document}
	\pagenumbering{gobble}
	\maketitle
	\newpage
	\tableofcontents
	\newpage
	\pagenumbering{arabic}	
	\section{Supported Languages}
		
		\begin{table}[h!]
			\begin{center}
				\caption{Officially supported languages by various tools}
				\label{tab:languageSupport}
				\begin{tabular}{ccccccccccccccc}
					\toprule
					Tool & Java & Go & C & C\verb!++! & C\verb!#! & Python & Perl & others\\
					\midrule
					Moss & \checkmark & & \checkmark & \checkmark & \checkmark & \checkmark & \checkmark & \checkmark \\
					JPlag & \checkmark & & \checkmark & \checkmark & \checkmark & & & \checkmark\\
					dupl & & \checkmark & & & & & & \\
					\bottomrule
				\end{tabular}
			\end{center}
		\end{table}
		

	\section{Getting the tools}
			
		\subsection{Moss}
			Moss is a web service, while JPlag and dupl can be run locally. To access Moss, one must send an email to \verb|moss@moss.stanford.edu| containing the following, with \verb|<email address>| replaced with one's actual email address:
			\begin{verbatim}
				registeruser
				mail <email address>
			\end{verbatim}
			\noindent Moss will then send a Moss upload script with a unique user ID, which should be placed in directory \verb|<x>|.
			
		\subsection{dupl}
			To download and install dupl, run the following command:
			\begin{lstlisting}[language=bash]
	go get -u github.com/mibk/dupl
			\end{lstlisting}
			\noindent dupl requires Go version 1.4 or higher.
			
		\subsection{JPlag}
			To download and install JPlag, get the code from \\\verb|https://github.com/jplag/jplag|. Maven is also required. Download and install it if necessary. Go to the \verb|jplag/jplag/jplag| directory inside the download and run the following command:
			\begin{lstlisting}[language=bash]
	mvn clean generate-sources assembly:assembly
			\end{lstlisting}
			This should create a jar file inside the \verb|./target/| directory called \\\verb|jplag-x.y.z-SNAPSHOT-jar-with-dependencies.jar| where \verb|x.y.z| is the specific version number of JPlag.
			
	\section{Tool commands}
		While it is not necessary to know the commands each anti-plagiarism tool uses, it may be helpful. The option described here are not the only options the tools use, but they are the options used by the application.
		
		\subsection{Moss}
			Here is an example of a Moss command:
			\begin{lstlisting}[language=bash, breaklines=true]
./moss -l java -m 2 -d ./code/class01/student01/assignment01/*.java ./code/class01/student02/assignment01/*.java ./code/class01/student03/assignment01/*.java > assignment01.txt &
			\end{lstlisting}
			\noindent The first argument is the Moss upload script. The $-l$ flag signifies that the next argument will be the language the assignments were written in, which in this case is Java. The $-m$ flag signifies that the following argument will be the threshold for Moss, which tells Moss to ignore matches that appear in more than this number of files. In this example, if a piece of code appears in more than 2 files, it is ignored. This is useful if instructors provide some functions or classes for their students to use. The $-d$ option signifies that directories will be compared instead of specific files. In this example, all the java files from three students' assignment 1 will be compared. Moss will search inside subdirectories. Finally the output from Moss is sent to a text file. The text file will contain a URL that has the results from Moss.
			
		\subsection{dupl}
			Here is an example of a dupl command:
			\begin{lstlisting}[language=bash, breaklines=true]
dupl -t 15 -html ./code/class01/student01/assignment02/ ./code/class01/student02/assignment02/ ./code/class01/student03/assignment02/ > assignment02.html &
			\end{lstlisting}
			\noindent The first argument is a call to dupl. The next argument, $-t$ is dupl's threshold. This is minimum nodes that pieces of code must be before dupl declares them as a duplicates. In this example, it is 15 nodes. $-html$ specifies html output. Next is a list of the directories. dupl will search inside subdirectories. Finally the output from dupl will be sent to an html file.
			
		\subsection{JPlag}
			Here is an example of a JPlag command:
			\begin{lstlisting}[language=bash, breaklines=true]		
java -jar ./jplag/jplag.jar -l java17 -t 15 -r ./results/lab1 -s lab1 ./students
			\end{lstlisting}
			\noindent The first three arguments say to run a Java jar file in that location called jplag.jar. $-l$ says which language to use, which in this case is Java 1.7. $-t$ is the minimum number of tokens to match (threshold) argument. The next argument, $-r$ specifies where to save the results. $-s$ says to check all the files in subdirectories with that label. So here any Java files in subdirectories labeled lab1 will be checked. The last argument is the base directory where the code resides.
			
	\section{Configuration file}
		There is a configuration file located in the base directory of the anti-plagiarism project called $config.txt$. LAB\_FILES\_BASE\_DIRECTORY tells the application where to store the code pulled from GitHub. MOSS\_FULLY\_QUALIFIED\_NAME contains the fully-qualified name of the Moss script. JPLAG\_FULLY\_QUALIFIED\_NAME contains the fully-qualified name of the JPlag jar file. RESULTS\_DIRECTORY tells the application where to store the results. MOSS\_THRESHOLD is the Moss threshold described in the Tool commands subsection. DUPL\_THRESHOLD and JPLAG\_THRESHOLD contain similar values.
			
		Here is an example:
		\begin{lstlisting}
LAB_FILES_BASE_DIRECTORY=/home/autograde/repos
MOSS_FULLY_QUALIFIED_NAME=/home/autograde/moss/moss
JPLAG_FULLY_QUALIFIED_NAME=/home/autograde/go/src/
	github.com/jplag/jplag/jplag/target/jplag.jar
RESULTS_DIRECTORY=/home/autograde/results
MOSS_THRESHOLD=4
DUPL_THRESHOLD=15
JPLAG_THRESHOLD=15
		\end{lstlisting}		

	\section{Process}
		The anti-plagiarism application will have two main functions. The first is to call the various anti-plagiarism detection tools, and the second is to check and store the results. It can take an indefinite amount of time for the tools to complete their analysis of the students' code, so it is best for the application to run the commands for the tools as a background process by using the $\&$ symbol at the end of each command.
		
		\subsection{Sending the commands and code}
			
		\subsection{Checking and storing the results}
\end{document}